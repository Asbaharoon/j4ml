\documentclass[12pt]{article}
\usepackage{graphicx}
\usepackage{amsmath}
\usepackage[margin=1.0in]{geometry}
\usepackage{color, colortbl}
\usepackage{neuralnetwork}
\usepackage{hyperref}
\usepackage{float}
\usepackage[affil-it]{authblk}
\usepackage[english]{babel}
\definecolor{LightCyan}{rgb}{0.88,1,1}
\definecolor{LightRose}{rgb}{1,0.88,0.88}
\definecolor{LightGreen}{rgb}{0.88,1,0.88}

\title{De-Noising Drift Chambers using Auto-Encoders for CLAS12}

\author[1]{Gagik Gavalian\thanks{Corresponding author}}
\affil[1]{Thomas Jefferson National Accelerator Facility, Newport News, VA 23606}
\date{}

\begin{document}

\begin{titlepage}
\maketitle
\begin{abstract}
In this article we describe the development of machine learning models to assist the CLAS12 tracking algorithm
by identifying tracks through inferring missing segments in the drift chambers.  Auto encoders are used to reconstruct 
missing segments from track trajectory. Implemented neural network was able to reliably reconstruct missing segment positions with accuracy of $\approx 0.35$ wires, and lead to recovery of missing tracks with accuracy of $>99.8\%$. 
\end{abstract}
\end{titlepage}

\section{Introduction}

\indent

The CLAS12\cite{Burkert:2020akg} detector is built around a six-coil toroidal magnet which divides the active detection into six azimuthal regions, called "sectors". The torus coils are approximately planar. Each sector subtends an azimuthal range of 60$^\circ$ from the mid-plane of one coil to the mid-plane of the adjacent coil. The “sector mid-plane” is an imaginary plane which bisects the sector’s azimuth. 
Charged particles in the CLAS12 detector are tracked using drift chambers\cite{Mestayer:2020saf} inside the toroidal magnetic field. There are six identical independent drift chamber systems in CLAS12 (one for each azimuthal sector). Each sector of drift chambers consists of three chamber sets (called "regions"), and each region consists of two chambers called super-layers each of them containing 6 layers of wires perpendicular to particle trajectories in CLAS12. The tracks passing through drift chambers leave signal in each of the layers (36 in total), which are broken down into 6 segments (one segment per super-layer). The tracking algorithm relies on forming track candidates on all combinations of segments (one from each super-layer).
With time some inefficiencies in detector develop leading to missing segments in one (or more) of the super-layers, and this results in efficiency drop in track identification. In this work we investigate neural networks that can help as improve the track finding efficiency when there are missing segments in some parts of the drift chambers.

\section{Reconstruction Procedure}

\indent

\newpage
\bibliography{references}
\bibliographystyle{ieeetr}

\end{document}
